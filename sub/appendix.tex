\section{부록}

\subsection{AWS Python 코드}

\lstset{basicstyle=\scriptsize, tabsize=4, numbers=left, keywordstyle=\color{blue}, commentstyle=\color{magenta}}

\begin{lstlisting}[language=python]
# Raspberry Pi ADC Python Code

# 0. Modules to import
import time
import Adafruit_DHT
import smbus
import pymysql
from gpiozero import Button

rain_sensor = Button(6)
BUCKET_SIZE = 0.2794
count = 0
count15 = 0
count60 = 0
count3h = 0
count6h = 0
count12h = 0
count1d = 0
i = 0
now15 = time.time()
now60 = time.time()
now3h = time.time()
now6h = time.time()
now12h = time.time()
now1d = time.time()
rain15 = 0
rain60 = 0
rain3h = 0
rain6h = 0
rain12h = 0
rain1d = 0
localcode = 111111
localname = 'Suwon'
temp = 0
airp = 0
rh = 0
# 1. Settings

f = open("windspeed.txt", 'r');
conn = pymysql.connect(host='ess.gs.hs.kr', user='AWS', password='rudrlrhkgkrrh',
db='AWS_GSHS', charset='utf8mb4')

curs = conn.cursor()
sql = """insert into 1min_data(OBS_code, OBS_datetime, preci_now, preci_15, preci_60, preci_3H, preci_6H, preci_12H, preci_1day, temperature, wind_spd01, RH, Air_P, REMARK)
values (%s, %s, %s, %s, %s, %s, %s, %s, %s, %s, %s, %s, %s, %s)""" # server database information

while True:

# 2. Humidity & Temperature
sensor_name = Adafruit_DHT.DHT22
sensor_pin = 25
humidity, temperature = Adafruit_DHT.read_retry(sensor_name, sensor_pin)
print('Temp = %.2f C' % temperature)
print('Humid = %.2f' % humidity)
temp = round(temperature, 2)
rh = round(humidity, 2)
time.sleep(1)

# 3. Pressure
bus = smbus.SMBus(1)
bus.write_byte(0x77, 0x1E)

time.sleep(0.25)

data = bus.read_i2c_block_data(0x77, 0xA2, 2)
C1 = data[0] * 256 + data[1]
data = bus.read_i2c_block_data(0x77, 0xA4, 2)
C2 = data[0] * 256 + data[1]
data = bus.read_i2c_block_data(0x77, 0xA6, 2)
C3 = data[0] * 256 + data[1]
data = bus.read_i2c_block_data(0x77, 0xA8, 2)
C4 = data[0] * 256 + data[1]
data = bus.read_i2c_block_data(0x77, 0xAA, 2)
C5 = data[0] * 256 + data[1]
data = bus.read_i2c_block_data(0x77, 0xAC, 2)
C6 = data[0] * 256 + data[1]
bus.write_byte(0x77, 0x40)

time.sleep(0.25)

value = bus.read_i2c_block_data(0x77, 0x00, 3)
D1 = value[0] * 65536 + value[1] * 256 + value[2]
bus.write_byte(0x77, 0x50)

time.sleep(0.25)

value = bus.read_i2c_block_data(0x77, 0x00, 3)
D2 = value[0] * 65536 + value[1] * 256 + value[2]
dT = D2 - C5 * 256
TEMP = 2000 + dT * C6 / 8388608
OFF = C2 * 65536 + (C4 * dT) / 128
SENS = C1 * 32768 + (C3 * dT) / 256
T2 = 0
OFF2 = 0
SENS2 = 0
if TEMP >= 2000:
T2 = 0
OFF2 = 0
SENS2 = 0
elif TEMP < 2000:
T2 = (dT * dT) / 2147483648
OFF2 = 5 * ((TEMP - 2000) * (TEMP - 2000)) / 2
SENS2 = 5 * ((TEMP - 2000) * (TEMP - 2000)) / 4
if TEMP < -1500:
OFF2 = OFF2 + 7 * ((TEMP + 1500) * (TEMP + 1500))
SENS2 = SENS2 + 11 * ((TEMP + 1500) * (TEMP + 1500)) / 2
TEMP = TEMP - T2
OFF = OFF - OFF2
SENS = SENS - SENS2
pressure = ((((D1 * SENS) / 2097152) - OFF) / 32768.0) / 100.0
cTemp = TEMP / 100.0
fTemp = cTemp * 1.8 + 32

print("Pressure : %.1f mbar" % pressure)
airp = round(pressure, 1)
time.sleep(0.25)


# 4. Rainfall
def bucket_tipped():
global count15
global count60
global count3h
global count6h
global count12h
global count1d
count15 += 1
count60 += 1
count3h += 1
count6h += 1
count12h += 1
count1d += 1


rain_sensor.when_pressed = bucket_tipped

if (time.time() - now15 >= 900):
rain15 = round(count15 * BUCKET_SIZE, 1)
count15 = 0
now15 = time.time()
if (time.time() - now60 >= 3600):
rain60 = round(count60 * BUCKET_SIZE, 1)
count60 = 0
now60 = time.time()
if (time.time() - now3h >= 10800):
rain3h = round(count3h * BUCKET_SIZE, 1)
count3h = 0
now3h = time.time()
if (time.time() - now6h >= 21600):
rain6h = round(count6h * BUCKET_SIZE, 1)
count6h = 0
now6h = time.time()
if (time.time() - now12h >= 43200):
rain12h = round(count12h * BUCKET_SIZE, 1)
count12h = 0
now12h = time.time()
if (time.time() - now1d >= 86400):
rain1d = round(count1d * BUCKET_SIZE, 1)
count1d = 0
now1d = time.time()
if (count15 >= 5):
count = 'O'
else:
count = 'X'
print(count)

# 5. Wind speed
line = f.readline()
wspeed = line
print(wspeed)

# 6. Time Test
print(time.time() - now15)

# 7. Send info to Database
k = time.strftime('%Y%m%d-%H%M%S', time.localtime(time.time()))
curs.execute(sql, (
localcode, time.strftime('%Y%m%d-%H%M%S', time.localtime(time.time())), count, rain15, rain60, rain3h, rain6h,
rain12h, rain1d, temp, wspeed, rh, airp, 'RnE')) # send data to server
g = open("aws.csv", 'a')
data = """%d, %s, %c, %f, %f, %f, %f, %f, %f, %f, %f, %f, %f\n""" % (
localcode, k, count, rain15, rain60, rain3h, rain6h, rain12h, rain1d, temp, float(wspeed), rh, airp)
g.write(data) # save data in raspberry pi
g.close()
conn.commit()
\end{lstlisting}

\subsection{풍속센서 작동을 위한 C언어 코드}

\lstset{basicstyle=\scriptsize, tabsize=4, numbers=left, keywordstyle=\color{blue}, commentstyle=\color{magenta}}

\begin{lstlisting}[language = c++]
#include <stdio.h>
#include <string.h>
#include <errno.h>
#include <unistd.h>
#include <wiringPi.h>
#include <wiringPiSPI.h>

#define CS_MCP3208  6        // BCM_GPIO 25

#define SPI_CHANNEL 0
#define SPI_SPEED   1000000  // 1MHz


int read_mcp3208_adc(unsigned char adcChannel)
{
unsigned char buff[3];
int adcValue = 0;

buff[0] = 0x06 | ((adcChannel & 0x07) >> 7);
buff[1] = ((adcChannel & 0x07) << 6);
buff[2] = 0x00;

digitalWrite(CS_MCP3208, 0);  // Low : CS Active

wiringPiSPIDataRW(SPI_CHANNEL, buff, 3);

buff[1] = 0x0F & buff[1];
adcValue = ( buff[1] << 8) | buff[2];

digitalWrite(CS_MCP3208, 1);  // High : CS Inactive

return adcValue;
}


int main (void)
{
FILE *out;
int adcChannel = 0;
int adcValue   = 0;

if(wiringPiSetup() == -1)
{
fprintf (out, "Unable to start wiringPi: %s\n", strerror(errno));
return 1 ;
}

if(wiringPiSPISetup(SPI_CHANNEL, SPI_SPEED) == -1)
{
fprintf (out, "wiringPiSPISetup Failed: %s\n", strerror(errno));
return 1 ;
}

pinMode(CS_MCP3208, OUTPUT);
int i = 1;
for(i=1; i<=3600; i++)
{
out = fopen("windspeed.txt", "a");
adcValue = read_mcp3208_adc(adcChannel);
if(adcValue <= 123) fprintf(out, "0\n"); // too weak voltage
else fprintf(out, "%.2f\n", (float)(20.25*adcValue)/1000)-8.1;
fclose(out);
sleep(2.54856); // sync with Python code
}


return 0;
}
\end{lstlisting}