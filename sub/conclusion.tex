\section{결론}

본 연구에는 라즈베리파이와 각종 센서들을 활용하여 AWS를 제작하고 그 자료를 서버 데이터베이스에 전송하였다. 연구 결과 대부분의 센서가 기상 정보를 정상적으로 수집하였으나 그렇지 않은 센서도 존재하였다. 

기압 센서는 통신 방법에 따라 비정상적인 값이 표시되기도 하였으며, 미세먼지 센서의 경우 원인 불명의 이유로 인해 프로그램에 오류가 생겨 자료 수집이 계속 중단되었다. 

풍향 센서 역시 풍향계의 방향과 무관하게 전압이 일정하게 표현되어 정보를 수집할 수 없었다. 그렇기에 추후 센서를 라즈베리파이에 연결하는 방법이나 프로그램을 변경해야 할 것으로 보이며 필요시 센서의 종류를 바꾸는 일도 필요해 보인다. 

또한 센서에서 표출한 값과 실제 값이 실제로 일치하는지에 대한 확인이 필요하며 만약 차이가 보인다면 보정식을 도출하여 센서에서 수집한 값을 보정해 주어야 할 것이다. 박종서, 오완탁 (1997)의 연구에서는 자동기상관측장비와 백엽상에서 수집한 기온 값의 차이가 0.1\~{}0.4$\degree$C 정도로 측정되었으며, 하지훈, 김용혁, 임효혁, 최덕환, 이용희 (2016)는 회귀분석을 사용해 자동기상관측장비의 기압자료를 보정하는 기법을 제시하였다\cite{Ref3}\cite{Ref4}. 

수집한 자료는 데이터베이스에 정상적으로 표출되었으며 이를 통해 장시간 관측으로 인해 생긴 많은 데이터를 체계적으로 관리할 수 있었으며 데이터 유실이나 훼손 등의 문제도 일어나지 않을 것으로 생각된다. 다만 기상청 날씨누리에서 제공하는 AWS 관측 자료와 비교하였을 때 본 장비에서는 수집되지 않는 기상 정보가 더 많았기 때문에 전문적인 기상 관측 자료로는 사용할 수 없으나, 온습도나 해면기압, 현재 강수 여부 등의 간단한 기상 정보는 비교적 쉽게 확인할 수 있다는 점에서 실용성이 있다고 생각된다. 

본 연구에서 개발한 AWS는 오류가 발생하거나 센서가 문제가 발생하였을 경우 비교적 쉽게 문제를 해결할 수 있고 센서의 가격 역시 비교적 저렴하기 때문에 AWS 장치 제작 이후의 관리 및 유지보수도 다른 AWS 보다 쉬울 것으로 판단된다. 또한 데이터베이스의 정보를 송죽학사 등의 교내 사이트에 정기적으로 전송해 교내 학생들이 언제나 기상 정보를 쉽고 편리하게 확인할 수 있게 하는 작업이 가능할 것으로 기대된다.